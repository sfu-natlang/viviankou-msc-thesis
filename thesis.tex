\documentclass{sfuthesis}
\title{Fast and Accurate Neural Network Models for Sequence Tagging}
\thesistype{Thesis}
\author{Xinxin Kou}
\previousdegrees{%
	B.Sc. (Hons.), Dalhousie University, 2015}
\degree{Master of Science}
\discipline{Computing Science}
\department{School of Computing Science}
\faculty{Faculty of Applied Sciences}
\copyrightyear{2017}
\semester{Fall 2017}
\date{12 September 2017}

\keywords{Natural Language Processing;}


%   PACKAGES AND CUSTOMIZATIONS  %%%%%%%%%%%%%%%%%%%%%%%%%%%%%%%%%%%%%%%%%%%%%%
%
%   Add any packages or custom commands you need for your thesis here.
%   You don't need to call the following packages, which are already called in
%   the sfuthesis class file:
%
%   - appendix
%   - etoolbox
%   - fontenc
%   - geometry
%   - lmodern
%   - nowidow
%   - setspace
%   - tocloft
%
%   If you call one of the above packages (or one of their dependencies) with
%   options, you may get a ''Option clash'' LaTeX error. If you get this error,
%   you can fix it by removing your copy of \usepackage and passing the options
%   you need by adding
%
%       \PassOptionsToPackage{<options>}{<package>}
%
%   before \documentclass{sfuthesis}.
%
\usepackage{natbib}
\usepackage{apalike}
\usepackage{amsmath,amssymb,amsthm}
\usepackage[pdfborder={0 0 0}]{hyperref}
\usepackage{graphicx}
\usepackage{caption}
%\usepackage[numbers]{natbib}
\usepackage{algorithm}
\usepackage{algpseudocode}
\usepackage{array}
\usepackage{multirow}
\usepackage{underscore}


\newcommand{\quotes}[1]{\textrm{``#1''}}

%   FRONTMATTER  %%%%%%%%%%%%%%%%%%%%%%%%%%%%%%%%%%%%%%%%%%%%%%%%%%%%%%%%%%%%%%
%
%   Title page, committee page, copyright declaration, abstract,
%   dedication, acknowledgements, table of contents, etc.
%

\begin{document}

\frontmatter
\maketitle{}
\makecommittee{}

\begin{abstract}
Sequence Tagging including part of speech tagging, chunking, named entity recognition is an important task in NLP. The recurrent neural network models such as the BiLSTM-CRF model have produced impressive results on sequence tagging. In this work, we present variant simple and fast feed-forward neural network models for use in greedy sequence tagging tasks. Besides the model of a simple feed-forward neural network with a small number of features, we provide a novel model using byte pair encoding on words for part-of-speech tagging, and a multitask model separating boundary and tag prediction for chunking and named entity recognition. We carefully design the experiments to show the relationship between speed and accuracy while using different models. Our experiment results show that the variants of feed-forward neural network models can achieve comparable accuracies and faster speed than the recurrent models on sequence tagging.
\end{abstract}


\begin{acknowledgements} % optional

I would like to show my appreciation to my supervisor Dr.\ Anoop Sarkar for the continuous support of my Masters study and research, for his patience, motivation, enthusiasm, and immense knowledge. His guidance helped me all the time, during the research and writing of this thesis. I could not have imagined having a better advisor and mentor for my Masters studies.

Thanks to all of my natural language processing lab mates who helped me during these two years and I really enjoyed being with them. 


\end{acknowledgements}

\addtoToC{Table of Contents}\tableofcontents\clearpage
\addtoToC{List of Tables}\listoftables\clearpage
\addtoToC{List of Figures}\listoffigures





%   MAIN MATTER  %%%%%%%%%%%%%%%%%%%%%%%%%%%%%%%%%%%%%%%%%%%%%%%%%%%%%%%%%%%%%%
%
%   Start writing your thesis --- or start \include ing chapters --- here.
%

\mainmatter%

\chapter{Introduction}


\section{Sequence Tagging Task}

\section{Dataset}

\section{Motivation}


\section{Contribution}


%Due to the lack of malicious chat messages, we parse the Wiktionary dataset and get the English sentences with offensive or normal tags as the dataset. We simulate the way users encode the messages and then we use Expectation Maximization (~\citeauthor{Dempster:77} ~\citeyear{Dempster:77}) and beam search to decode the most likely original message. 

\section{Overview}
The thesis is organized as follows:

In \textbf{Chapter 2}  we present variants of Feedforward Neural Network Models, explain the experiments design, and show the experiments results.

In \textbf{Chapter 3} 

In \textbf{Chapter 4} we present variants Bi-LSTM Network Models, explain the experiments design, and show the experiments results.

In \textbf{Chapter 5} we present a multitask Model based on the Mention2Vec Model, explain the experiments design, and show the experiments results.

In \textbf{Chapter 6} we summarize the experimental results of the previous chapter.

\chapter{Feedforward Neural Network Models}
\section{Model Description}
\section{Part-of-Speech Experiments and Results}
\section{Named Entity Recognition Experiments and Results}

\chapter{Bi-direction Long Short Term Memory Network Models}
\section{Model Description}
\section{Part-of-Speech Experiments and Results}
\section{Named Entity Recognition Experiments and Results}

\chapter{Mention2Vec Model}
\section{Model Description}
\section{Part-of-Speech Experiments and Results}
\section{Named Entity Recognition Experiments and Results}

\chapter{Conclusion}

\section{Experimental Design}
\begin{table}[]
\centering
\caption{Neural Network Models Accuracy and F-Score}
\label{my-label}
\begin{tabular}{|c|c|c|}
\hline
Model         & POS (Accuracy)  & NER (F-Score)       \\ \hline
BI-LSTM  & 95.9     & 84.7                             \\ \hline
BI-LSTM-CRF & \textbf{97.4} & \textbf{90.1}                             \\ \hline
Feed-Forward    & 95.8          &   84.1                                         \\ \hline
Feed-Forward-CRF & 97.3     & 86.4                          \\ \hline
Mention2Vec & _    & 87.8                         \\ \hline
BPE-Mention2Vec  &     &  _   \\ \hline   
\end{tabular}
\end{table}


\section{Experimental results}


\chapter{Conclusion}


%   BACK MATTER  %%%%%%%%%%%%%%%%%%%%%%%%%%%%%%%%%%%%%%%%%%%%%%%%%%%%%%%%%%%%%%
%
%   References and appendices. Appendices come after the bibliography and
%   should be in the order that they are referred to in the text.
%
%   If you include figures, etc. in an appendix, be sure to use
%
%       \caption[]{...}
%
%   to make sure they are not listed in the List of Figures.
%

%\backmatter%
\cleardoublepage
\phantomsection
\addtoToC{Bibliography}
%\bibliographystyle{apacite}
\bibliographystyle{apalike}
\bibliography{references}
	

%\begin{appendices} % optional
%	\chapter{Code}
%\end{appendices}
\end{document}
